\documentclass{article}

\usepackage{mathtools}
\usepackage{amssymb}
\usepackage{amsmath}

\title{Dytype: a strongly normalizing, almost-dynamically-typed Lisp}
\author{Keith Pinson}

\begin{document}

\maketitle

For any primitive non-function literal $c$, we can synthesize its type as $1$
from the empty context:

\[ \frac{}{\vdash{}c\mathbin{\Rightarrow}1}\quad\text{(literal synthesis)} \]

Our only hope of synthesizing a lambda is to try and see if we can synthesize
something meaningful if we assign the parameter type $1$ (unless I read up more
on Hindley-Milner and better understand how to do this).

\[ \frac{\Gamma,x\mathbin{:}1\vdash e\mathbin{\Rightarrow}\alpha}{\Gamma\vdash
    (\lambda~(x)~e)\mathbin{\Rightarrow}1\to\alpha}\quad\text{(unary
    $\lambda$ synthesis)} \]

This can easily be extended to forms like $(\lambda~(x~y~z)~e)$ with
currying (or without, if we want to fully concede to Lisp).

Variadic functions can be recognized syntactically. Essentially, they just take
in the \textbf{cdr} when they are applied, which is just data, but needs a
sort of modal distinction, so

\[ \frac{\Gamma,x\mathbin{:}1\vdash e\mathbin{\Rightarrow}\alpha}{\Gamma\vdash
    (\lambda~x~e)\mathbin{\Rightarrow}\star\to\alpha}\quad\text{(variadic
    $\lambda$ synthesis)} \]

We have a bit of syntactic sugar for function types, based on reducing to
naturals numbers:

\[ 1\to1 = 2 \]
\[ 1\to2 = 3 \]

and so on. Only plain first-order functions can be reduced to a natural;
higher order functions reduce like so:

\[ (1\to{}1)\to{}1\to{}1 = 2\to{}2 \]

Notably, this does not reduce further:

\[ 2\to{}2 \not = 4 \]

and variadic functions don't reduce to a natural, either, e.g.:

\[ \star\to{}1\to{}1 =
  \star\to{}2 \]

A variadic function of type like
$1\to\star\to{}1$ does not reduce. Also, in
Lisp we don't have the nice infix arrow notation, so we actually write a type
like $2\to{}2$ as $(2~.~2)$ and $1\to\star\to{}1$ as $(1~.~*~.~1)$.

The application rules are mundane:

\[ \frac{\Gamma\vdash f\mathbin{\Rightarrow}\alpha\to\beta\quad\Gamma\vdash
    x\mathbin{\Leftarrow}\alpha}{\Gamma\vdash
    (f~x)\mathbin{\Rightarrow}\beta}\quad\text{(Unary application synthesis)} \]

\[ \frac{\Gamma\vdash f\mathbin{\Rightarrow}\star\to\alpha\quad\Gamma\vdash
    x\mathbin{\Leftarrow}1}{\Gamma\vdash
    (f~.~x)\mathbin{\Rightarrow}\alpha}\quad\text{(Variadic application synthesis)} \]

The second rule's syntax is imprecise: I do not mean that a literal dotted pair
is the syntax for invoking, but rather that the entire \textbf{cdr} is captured
and passed to the function. In other words, $(f~a~b~c)$ for the second rule is
treated like $(f~(\textbf{list}~a~b~c))$ would be for the first rule. This is a
concession to Lisp, rather than a particularly salient point of Dytype itself,
as I see it.

We adopt the standard rule that allows checking to be far more powerful via
synthesis:

\[ \frac{\Gamma\vdash e\mathbin{\Rightarrow}\alpha}{\Gamma\vdash
    e\mathbin{\Leftarrow}\alpha}\quad\text{(Synthesis suffices for checking)} \]

And we allow for explicit annotations:

\[ \frac{\Gamma\vdash e\mathbin{\Leftarrow}\alpha}{\Gamma\vdash
    (: e~\alpha)\mathbin{\Rightarrow}\alpha}\quad\text{(Annotation synthesis)} \]

We do not want to implement a separate totality checker when the entire point of
the type-system is to provide totality. At the same time, the rules we have
provided do not offer any form of recursion whatsoever; the power of what is
already presented, aside from whatever we may decide to surface from the
underlying Scheme in the Prelude, is insufficient for almost any interesting
real coding example (fizzbuzz, anyone?). We want the language to be sufficiently
strong to typecheck itself; ideally, I would like to be able to do the Lisp
thing and write Dytype interpreter in Dytype (but am not sure if this will be
achievable). So we need rules for recursion, that allow us a sufficiently useful
amount of recursion, and yet rigorously enforce totality.

We propose a new binding form, $\mu$, that allows defining two mutually
recursive functions (how this can be simplified to the case of a self-recursive
function, or extended to a larger set of mutually recursive functions, or
extended to functions of multiple arguments, or extended to \textbf{cadr},
\textbf{cddr} and the like, or to numbers, is all contained by way of exemplar
in these rules). The first function, if it calls itself, must call itself back
with a substructure, and also if it calls the second. The second, if it calls
itself back, must use a substructure; but it may call the first with the same
size argument it received. The utility of this little extra bit of freedom is
seen when one tries to implement a typechecker that includes a rule like
``Synthesis suffices for checking'', above.

%% TODO this looks pretty bad
\[
  \frac{
    \splitfrac{
      \splitfrac{
        \Gamma,x\mathbin{:}x^{=},f_{1}\mathbin{:}x^{>}\to\alpha_{1},f_{2}\mathbin{:}x^{>}\to\alpha_{2}\vdash e_{1}\mathbin{\Leftarrow}\alpha_{1}
      }{
        \Gamma,x\mathbin{:}x^{=},f_{1}\mathbin{:}x^{\geq}\to\alpha_{1},f_{2}\mathbin{:}x^{>}\to\alpha_{2}\vdash e_{2}\mathbin{\Leftarrow}\alpha_{2}
      }
    }{
      \Gamma,f_{1}\mathbin{:}1\to\alpha_{1},f_{2}\mathbin{:}1\to\alpha_{2}\vdash e_{3}\mathbin{\Rightarrow}\alpha_{3}
    }
  }{
    \Gamma\vdash
    (\mu~f_{1}~(x)~e_{1}~(1~.~\alpha_{1})~f_{2}~(x)~e_{2}~(1~.~\alpha_{2})~e_{3})\mathbin{\Rightarrow}\alpha_{3}
  }
\]

The expression is to be read as, ``define two mutually recursive functions,
$f_{1}$ and $f_{2}$, on recursive variable $x$, with function bodies $e_{1}$ and
$e_{2}$, respectively, and types $1\to\alpha_{1}$ and
$1\to\alpha_{2}$, respectively, such that they are bound in
expression $e_{3}$''. I have written this as a binding form, because the typing
rules are best expressed that way; in practice as implemented in a Scheme you
may want rather to do a conceptually similar thing as a top-level form, and
elaborate simply to \textbf{define}s.

For this to work properly, we have to do something a little goofy and disallow
shadowing on variables that are the subject of recursion: or perhaps a special
symbol could be reserved for such a variable. In other words, the appearance of
the variable in the type is not an indication of dependent type, but just of a
dirty little syntactic trick. I call this goofy, because I have not expressed it
fully in the formalism yet.

\[ \frac{\Gamma\vdash f\mathbin{\Rightarrow}x^{>}\to\alpha\quad\Gamma\vdash
    x\mathbin{\Rightarrow}x^{=}}{\Gamma\vdash
    (\downarrow~f~\textbf{car}~x)\mathbin{\Rightarrow}\alpha} \]

\[ \frac{\Gamma\vdash f\mathbin{\Rightarrow}x^{>}\to\alpha\quad\Gamma\vdash
    x\mathbin{\Rightarrow}x^{=}}{\Gamma\vdash
    (\downarrow~f~\textbf{cdr}~x)\mathbin{\Rightarrow}\alpha} \]

\[ \frac{\Gamma\vdash f\mathbin{\Rightarrow}x^{\geq}\to\alpha\quad\Gamma\vdash
    x\mathbin{\Rightarrow}x^{=}}{\Gamma\vdash
    (\downarrow~f~\textbf{car}~x)\mathbin{\Rightarrow}\alpha} \]

\[ \frac{\Gamma\vdash f\mathbin{\Rightarrow}x^{\geq}\to\alpha\quad\Gamma\vdash
    x\mathbin{\Rightarrow}x^{=}}{\Gamma\vdash
    (\downarrow~f~\textbf{cdr}~x)\mathbin{\Rightarrow}\alpha} \]

\[ \frac{\Gamma\vdash f\mathbin{\Rightarrow}x^{\geq}\to\alpha\quad\Gamma\vdash
    x\mathbin{\Rightarrow}x^{=}}{\Gamma\vdash
    (\downarrow~f~x)\mathbin{\Rightarrow}\alpha} \]

The semantics of these calls are such that $(\downarrow~f~\textbf{car}~x)$ would
translate to simply $(f~(\textbf{car}~x))$ in the Scheme backend, etc.

I do not have any formal proof (yet) of the soundness of this system, or its
termination. However, it appears to me to be simple enough so that with a little
reflection, it will be found convincing.

\end{document}

